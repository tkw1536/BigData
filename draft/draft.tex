\documentclass[a4paper,12pt]{article}

% UTF-8
\usepackage[utf8]{inputenc}

% Jinbo
\usepackage{hyperref}
\usepackage{graphicx}
\usepackage{float}

% BibTeX needs urls
\usepackage{booktabs}
\usepackage{hyperref}

\bibliographystyle{plain}

% Math - why not?
\usepackage{amsfonts}
\usepackage{amssymb}
\usepackage{amsmath}

% ednotes - please use them
\usepackage[show]{ed}

% make the title page a bit smaller
\usepackage{titling}

% Force single spacing
\renewcommand{\baselinestretch}{1.0}

\title{Big Data Initiative:\\Effective Caching in Online Video Platforms\\Draft\ednote{Remove draft status}}
\author{Rongrong Bao \and Atabak Hafeez \and Tom Wiesing \and Jinbo Zhang}
\date{\today}

\begin{document}

\maketitle

\begin{abstract}
  Data on the internet grows by 50 percent annually. More than 90\% of the data has been generated in recent years. This is the time for big data. How can we effectively transfer this huge amount of data?

  We want to investigate caching techniques used by online video platforms and in particular by YouTube. YouTube is a leading online video provider worldwide. Before 2012, video streaming in YouTube was done using Real Time Messaging Protocol (RTMP)-based servers. This requires a streaming server and a near-continuous connection between the server and user. Requiring such a streaming server can increase implementation cost and RTMP-based video streaming is at risk of being blocked by firewalls. In 2012, this was replaced by HTTP (Hypertext Transfer Protocol) based servers known as MPEG DASH (Dynamic Adaptive Streaming over HTTP). HTTP is the protocol used by websites to bring their content to the users. By using this technology it was possible to use existing optimizations in the form of HTTP-Caching. This capability decreased total bandwidth costs associated with delivering the video since videos would be served from web-based caches rather than the origin server. This improved quality of service, since cached data is generally closer to the viewer and more easily retrievable.

  The essay will explain and discuss different kinds of caching techniques, optimizations, data analysis and prediction techniques used by YouTube, including their advantages/disadvantages and potential social impacts.
\end{abstract}

\newpage

\tableofcontents

\newpage
\section{Introduction}
In this paper we will discuss YouTube -- the biggest video-sharing website in the world -- as a typical example of a video sharing platform that uses big data. We want to explain what caching is, how it is used in YouTube and want to discuss its advantages and disadvantages. YouTube has an average of one hour of video uploads per second, and an average of 35 million video uploads daily. According to \cite{expandedramblings:stats}, YouTube now has more than 1 billion users, which is almost a third of total daily consumption of the world's Internet video viewing time. As of this year, YouTube's viewing time has increased by 60\%. Without making use of Big Data Analytics, YouTube will not be able to keep up with the needs of the users in the world. 

But what is Big Data? This question has become more and more important in recent years. A possible answer for this is given by Viktor Mayer-Schoenberg in \cite{Mayer-Schnberger:2013:BDR:2588165} where he refers to the fact that Big Data does not just use random analysis but uses all available data for analysis and processing. This answer does not completely answer the question. 

The United States Internet Data Center pointed out that the data on the Internet will soon grow by 50 percent annually. During a single day the internet today already generates new content to fit on 168 million DVDs. 2 million community posts are made every day. This is the same as number of letters printed in 770 years of Time Magazine. The data produced by emails is much bigger. It fills 294 billion DVDs. As of 2012, the amount of data available on the Internet jumped from Terabyte (TB\footnote{1 TB is 1000 Gigabytes}) level via Petabyte (PB\footnote{1 PB is 1000 TB})) and Exabyte (EB\footnote{1 EB is 1000 TB}) level to the Zettabyte (ZB\footnote{1 ZB is 1000 EB}) level. The International Data Corporation (IDC) showed that by in 2020 the size of worldwide generated data will reach 44 times the amount of today. 

As we have seen from the examples above Big Data and Big Data Analysis are becoming more and more important. Real-time big data analytics can be of immense importance for a business. But a business must first evaluate pros and cons of this process as it is still a relatively new technology. It is expected to evolve in the future and hopefully resolve some of its current challenges. We will discuss this in detail for the example of YouTube in this paper. 

In section \ref{whatiscaching} we will first introduce what caching is and give some basics on how YouTube makes use of this. Afterwards in section \ref{advantages} we will focus on this further and elaborate the advantages of YouTubes caching. After discussing the disadvantages of caching in section \ref{disadvantages} we finally conclude in section \ref{conclusion} by reflecting on the points made. 


\section{What is Caching?}
Before we delve into the specifics of how YouTube does caching and the advantages and disadvantages of the different kinds of caching, we will explain what caching itself is and explain the different techniques used by YouTube. 

Caching is a method to store data in a cache. A cache is a basically temporary storage area on the local hard disk of a user. This storage area may contain data such as HTML (Hypertext Markup Language - a language to describe web documents or pages) pages, images, files, and web Projects in order to make it faster for the user to access it, which helps improve the efficiency of the computer and the overall efficiency of the task at hand. The important thing to note here is that it occurs mostly without the user being aware of exactly which data has been stored in the cache. For example, when a user returns to a web page they have recently accessed, the browser can pull those files from the cache instead of the original server because it has stored the user's activity. The storing of that information saves the user time by getting to it load faster, reduces local memory usage and lessens the traffic on the network. 


\subsection{Caching and Buffering in YouTube}
To explain caching in YouTube and how it has changed and updated, we also need to understand the notion of buffering. Buffering involves pre-loading data into a certain area of memory known as a ``buffer'' in the local machine. This is basically a a more specific kind of caching which YouTube uses to store the loaded video on to the local memory of the browser in use.

In 2013, YouTube made a design decision in their buffering system where they moved from Real Time Messaging Protocol (RTMP)-based Dynamic Streaming to MPEG DASH (Dynamic Adaptive Streaming over HTTP).

To a user, this is important because it changes the extent to which you can cache your YouTube video before viewing it. Basically, while YouTube was using RTMP-based Dynamic Streaming, if a user had a relatively slow connection, which would not allow them to view the video as smoothly as one would want, he/she could pause the video and view it later when the whole video is buffered or cached to the local storage of the browser. This technology required a near-continuous connection between the server - the original storage location where one's local computer is connected to retrieve the video - and the player on one's browser.

With YouTube's shift to using MPEG DASH, being able to buffer the whole video and then coming back to it was no longer possible. MPEG DASH uses standard HTTP (Hypertext Transfer Protocol - a set of standards which defines how messages are formatted and transmitted across the World Wide Web) web servers to deliver streaming content, obviating the need for a streaming server. In addition, HTTP packets are firewall (a set of programs that block unauthorised access to a computer) friendly and can utilise HTTP caching mechanisms on the web. To an average user, this means that now when he or she pauses a video because the video is not very smooth, the video buffers for a while and then stops buffering. The cache, hence, does not at this stage store the whole video. The only condition in which the video may cache the whole video is if you start the video and watch the whole way through. In this case, if the user turns off the internet, it would be possible for the user to re-watch the video without having to reload it from the server.

\subsection{Caching on the Servers}
So why did YouTube decide to make this significant change in their protocol? This is mainly because this makes streaming high quality videos more efficient by caching on the servers.

Caching on the server means that when a user connects to a server, the connection is not direct. There is something in-between known as a caching server. The caching server acts as a web proxy server so it can serve those requests. After a web proxy server receives requests for web objects, it either serves the requests or forwards them to the origin server (the web server that contains the original copy of the requested information). Using MPEG DASH, YouTube was able to exploit this feature resulting in higher resolution videos being available to the user in a more efficient manner.

This form of caching also aims to make sure that the user can experience the best quality of video according to the available bandwidth speed. For example, a video may start playing at 360P resolution, but if the system detects that the bandwidth is now able to handle 720P, it will shift to that. Thus, using MPEG DASH, YouTube only caches chunks of the video and only a small chunk of them are loaded when a video is paused.

\ednote{No disadvantages here?}


\section{Advantages of Caching}

Notably, YouTube used two techniques to deliver web contents effectively -- Distributed Caching and Content Delivery Network.

A distributed cache is an extension of the traditional concept of cache used in a single locale. A distributed cache may span multiple servers so that it can grow in size and in transactional capacity\ednote{\url{https://en.wikipedia.org/wiki/Distributed_cache}}. Before we talked about distributed caching technique more deeply, we introduce some terms to help us better understand the idea behind it. In communication networks, a \textbf{node} is either a connection point, a redistribution point, or a communication endpoint. Speaking of distributed network, the nodes are clients, servers or peers. By storing the data not on the individual web server's memory but on other cloud resources, \textbf{distributed cache} offers high throughput, low-latency access to commonly accessed application data.

One significant advantage for distributed caching, is that when the application scales by adding or removing servers, or when servers are replaced due to upgrades or faults, the cached data remains accessible to every server that runs the application. For example, If we have data \texttt{1, 2, 3, 4, 5} and servers \texttt{A, B, C} If we store \texttt{1, 2, 5} in \texttt{A}, \texttt{3, 5, 4} in B, \texttt{1, 2, 3, 4} in \texttt{C}. If one server is down, we won't lose any data, because every piece of data has a copy in another server.

\ednote{Graph needed here. Someone please?}

By distributing data efficiently and effectively(typically by different hashing techniques), caching can dramatically improve application responsiveness. Comparing to access data from relational databases, which requires a lot computations, accessing data from cache is much faster. Caching works best for application workloads that do more reading than writing of data, and when application users share a lot of common data. So this is why such technique is very useful for YouTube: those popular videos will be watched again and again.

\begin{figure}[H]
	\centering
	\includegraphics[width=\linewidth]{CallMeMaybe.png}
	\caption{A very popular video in YouTube which is watched more than 700 million times}
\end{figure}

In order to get data from cache, the data must exist in cache before retrieving. There are several strategies for putting data into a cache:\ednote{\url{http://www.asp.net/aspnet/overview/developing-apps-with-windows-azure/building-real-world-cloud-apps-with-windows-azure/distributed-caching}}

\textbf{On Demand / Cache Aside} The application tries to retrieve data from cache, and when the cache doesn't have the data (a "miss"), the application stores the data in the cache so that it will be available the next time. The next time the application tries to get the same data, it finds what it's looking for in the cache (a "hit").

\textbf{Background Data Push} Background services push data into the cache on a regular schedule, and the application always pulls from the cache. This approach works great with high latency data sources that don't require you always return the latest data.

\textbf{Circuit Breaker} The application normally communicates directly with the persistent data store, but when the persistent data store has availability problems, the application retrieves data from cache.

Strategy 2 is clearly the best for YouTube. YouTube will the count the watching times in order to filter the popular videos from non-popular ones, or count the watching times per minute in order to filter the trendy videos from the outdated ones, thus the background services have enough information to determine which videos will be pushed into cache. Also, once the video is uploaded, it won't change anymore, so the cached data will always stay tuned with the original one.

When we type \url{www.youtube.com} in our browser, How can they determine which server will respond our request? There are several different techniques for determining the responsive server, in the paper, we will mainly focus on one of the them -- Content Delivery Network.

A Content Delivery Network is a system of distributed servers (network) that deliver web pages and other Web content to a user based on the geographic locations of the user, the origin of the web page and a content delivery server \ednote{\url{http://www.webopedia.com/TERM/C/CDN.html}}.

This service is effective in speeding the delivery of content of websites with high traffic and websites that have global reach. The closer the CDN server is to the user geographically, the faster the content will be delivered to the user. CDNs also provide protection from large surges in traffic.

Servers nearest to the website visitor respond to the request. The CDN copies the pages of a website to a network of servers that are dispersed at geographically different locations, caching the contents of the page. When a user requests a webpage that is part of a content delivery network, the CDN will redirect the request from the originating site's server to a server in the CDN that is closest to the user and deliver the cached content. The CDN will also communicate with the originating server to deliver any content that has not been previously cached \ednote{These three part need to polish.}.

Thus, the CDN can reduce traffic on the primary network, which improves video content and web performance overall.


\section{Disadvantages of Caching}
While caching videos in the users web browser has many useful effects as discussed above it also has several bad side effects. In this section we will first discuss some of the technical side effects and then continue with some of the social implications\ednote{Better wording, maybe a longer introduction and link to the previous section}. 

\subsection{Technical side effects}

In order to take advantage of caching of videos the browser needs to create caches for all the videos that the user plays. Additionally every time a video is played the browser has to check if a cached version of the video already exists. If this is the case the video can be played from the cached version. If this is not the case however the browser has to either create a cached version of the video or play a direct version from the server. Both of these decisions have a negative performance impact since the video has to be downloaded from the server which is slower then playing a local version.

Furthermore a cached version might not always be up-to-date. It is conceivable that when a news station has uploaded a video to YouTube and later on finds an error in their report that they might update their video in order to provide the most correct and up-to-date information. If this video has been cached by a viewer of the video they might no longer have the newest version available in their cache. Thus in order to use caches properly every time the cache is used the web browser needs to check if it is still up-to-date. This process can be further complicated if only parts of the video are cached. Furthermore if the cache is invalidated (meaning it is found to no longer be up-to-date) it is inefficient to reload the entire video if only parts of it have changed.

Since caching technically copies the videos from the server to the hard disks of the local user the video is no longer stored in only one central location. This means that even if the video has to be deleted (for example for legal reasons) it might still be available locally. This might mean legal hurdles if a license has to be obtained in order to show the video.

Additionally some of the videos the user watches might remain cached on the users hard drive even after they have watched the video. While a single video does not take too much space this can cause a problem if there are many of these cached videos. It is especially important for mobile users as these typically have less space available on their devices. Furthermore the user might not want a record of the videos they watched for privacy reasons. 

\section{Conclusion}

\renewcommand\refname{\vskip -1cm}
\section{References}
\bibliography{draft}
\end{document}

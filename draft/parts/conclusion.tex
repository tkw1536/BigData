In this paper we have talked about caching techniques, how they are used by YouTube in particular, and they relate to Big Data. This is an era of data explosion, if we don't have such a substantial size of data, caching techniques won't be so essential to those online video platforms. We have discussed how caching enables the watching of videos on the Internet by downloading the video onto the hard disk of the local computer. We have mentioned how techniques like distributed caching or the concept of CDNs can increase user experiences from the technical side. Even though this brings a lot of advantages we have also seen how this can have negative side effects and other social implications.

These side effects are still open for discussing, Clearly there's no elixir for the problems, and they  cannot be solved by technical solutions alone. Social impacts need to be considered and, like almost every other use of Big Data, this involves taking privacy issues into account. In general (Big) Data collection always has good and bad effects. The disadvantages mostly come from privacy issues whereas the advantages primarily come from the fact that it can make technical solutions more efficient. 

It makes sense to let users decide, to offer options to the them. A controversial issue related to each individual should be solved by themselves -- after all, people are different from each other. They can decide if they want a better experience in their services or a better protection of their privacy.  

It also makes sense to improve the transparency of data usage. Probably everyone enjoys better experience, but some of them are more concerned about if they are monitored by someone or some institution. For the data collectors, if they can offer more concrete evidences, to show that the collected data are encrypted, are stored safely, and they won't be decrypted by any malicious people, or be transferred as other usages, then these movements can certainly reassure some worried people.



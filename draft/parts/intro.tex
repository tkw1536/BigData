In this paper we will discuss YouTube -- the biggest video-sharing website in the world -- as a typical example of a video sharing platform that uses big data. We want to explain what caching is, how it is used in YouTube and want to discuss its advantages and disadvantages. YouTube has an average of one hour of video uploads per second, and an average of 35 million video uploads daily. According to \cite{expandedramblings:stats}, YouTube now has more than 1 billion users, which is almost a third of total daily consumption of the world's Internet video viewing time. As of this year, YouTube's viewing time has increased by 60\%. Without making use of Big Data Analytics, YouTube will not be able to keep up with the needs of the users in the world. 

But what is Big Data? This question has become more and more important in recent years. A possible answer for this is given by Viktor Mayer-Schoenberg in \cite{Mayer-Schnberger:2013:BDR:2588165} where he refers to the fact that Big Data does not just use random analysis but uses all available data for analysis and processing. This answer does not completely answer the question. 

The United States Internet Data Center pointed out that the data on the Internet will soon grow by 50 percent annually. During a single day the internet today already generates new content to fit on 168 million DVDs. 2 million community posts are made every day. This is the same as number of letters printed in 770 years of Time Magazine. The data produced by emails is much bigger. It fills 294 billion DVDs. As of 2012, the amount of data available on the Internet jumped from Terabyte (TB\footnote{1 TB is 1000 Gigabytes}) level via Petabyte (PB\footnote{1 PB is 1000 TB})) and Exabyte (EB\footnote{1 EB is 1000 TB}) level to the Zettabyte (ZB\footnote{1 ZB is 1000 EB}) level. The International Data Corporation (IDC) showed that by in 2020 the size of worldwide generated data will reach 44 times the amount of today. 

As we have seen from the examples above Big Data and Big Data Analysis are becoming more and more important. Real-time big data analytics can be of immense importance for a business. But a business must first evaluate pros and cons of this process as it is still a relatively new technology. It is expected to evolve in the future and hopefully resolve some of its current challenges. We will discuss this in detail for the example of YouTube in this paper. 

In section \ref{whatiscaching} we will first introduce what caching is and give some basics on how YouTube makes use of this. Afterwards in section \ref{advantages} we will focus on this further and elaborate the advantages of YouTubes caching. After discussing the disadvantages of caching in section \ref{disadvantages} we finally conclude in section \ref{conclusion} by reflecting on the points made. 

In Viktor Mayer-Schoenberg's book \textit{Big Data: A Revolution That Transforms How we Work, Live, and Think} (with Kenneth Cukier), it refers that big data doesn't simply use random analysis (sampling) as a shortcut; instead, it uses all data analysis and processing. There are 4V features of big data: volume, velocity, variety, value.

The USIDC (United States Internet Data Center) pointed out that the data on the Internet will grow by 50 percent annually. More than 90\% of the data has been generated in recent years. In addition, the worldwide industrial equipment, such as automobiles, motion, vibration, temperature, and even the change of chemical substances in air humidity, also produced vast amounts of data.\ednote{Need polishing} Internet, cloud computing, mobile Internet, car networking, phone, tablet, PC and a variety of sensors are the sources of big data. The value of big data is in the following aspects:

\begin{enumerate}
  \item to provide products or service businesses that a large number of consumers can take advantage of big data precision marketing;
  \item some small companies or businesses can take advantage of big data to do service transformation;
  \item under the pressure of the traditional Internet, companies need to take advantage of the times of the value of big data.
\end{enumerate}

Real-time big data analytics can be of immense importance to a business, but a business must first determine if the pros outweigh the cons in their particular situation, and if so, how those cons will be overcome. This is still a relatively new technology, so it is expected to evolve in the future and hopefully resolve some of its current challenges.

How big is BIG DATA? Using only the Internet as an example, during a day, the entire contents of the Internet can be engraved to produce 168 million DVDs; the amount of emails sent are as much as 294 billion; the amount of community posts sent over is two million, same as the number of letters printed in 770 years of Time magazine. As of 2012, the amount of data available on the Internet already jumped from TB (1024GB = 1TB) level to PB (1024TB = 1PB), from EB (1024PB = 1EB) to the ZB (1024EB = 1ZB) level. The IDC (International Data Corporation) showed that by 2020, the size of data generated worldwide will reach 44 times the amount of today.

The Internet, obviously, is mostly based on and benefits from Big Data. What we are discussing in our project, YouTube is the biggest video-sharing website in the world. It has an average of one hour of video uploads per second, and an average of 35 million video uploads daily. According to the article 120+ amazing YouTube statistics,\ednote{Need references} it shows that YouTube now has more than 1 billion users, which is almost a third of total daily consumption of all the world Internet video viewing time. By 2015, the YouTube viewing time increased by 60\%, which is on the highest level of growth. Without the process of development of Big Data, YouTube will not be able to keep up with the needs of the users in the world.

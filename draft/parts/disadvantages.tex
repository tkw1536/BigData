While caching videos in the users web browser has many useful effects as discussed above it also has several bad side effects. In this section we will first discuss some of the technical side effects and then continue with some of the social implications\ednote{Better wording, maybe a longer introduction and link to the previous section}. 

\subsection{Technical side effects}

In order to take advantage of caching of videos the browser needs to create caches for all the videos that the user plays. Additionally every time a video is played the browser has to check if a cached version of the video already exists. If this is the case the video can be played from the cached version. If this is not the case however the browser has to either create a cached version of the video or play a direct version from the server. Both of these decisions have a negative performance impact since the video has to be downloaded from the server which is slower then playing a local version.

Furthermore a cached version might not always be up-to-date. It is conceivable that when a news station has uploaded a video to YouTube and later on finds an error in their report that they might update their video in order to provide the most correct and up-to-date information. If this video has been cached by a viewer of the video they might no longer have the newest version available in their cache. Thus in order to use caches properly every time the cache is used the web browser needs to check if it is still up-to-date. This process can be further complicated if only parts of the video are cached. Furthermore if the cache is invalidated (meaning it is found to no longer be up-to-date) it is inefficient to reload the entire video if only parts of it have changed.

Since caching technically copies the videos from the server to the hard disks of the local user the video is no longer stored in only one central location. This means that even if the video has to be deleted (for example for legal reasons) it might still be available locally. This might mean legal hurdles if a license has to be obtained in order to show the video.

Additionally some of the videos the user watches might remain cached on the users hard drive even after they have watched the video. While a single video does not take too much space this can cause a problem if there are many of these cached videos. It is especially important for mobile users as these typically have less space available on their devices. Furthermore the user might not want a record of the videos they watched for privacy reasons. 